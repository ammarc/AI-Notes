\documentclass[twoside]{article}
\usepackage{lmodern}
\title{Artificial Intelligence: Week 2}
\date{}
\author{}
\begin{document}
\maketitle
\section{Problem Solving by Searching}

\textbf{Goal formulation}, based on the current situation and the agent's
performance measure, is the first step in problem solving. \textbf{Problem 
formulation} is the process of deciding what actions and states to consider,
given a goal. In general, \emph{an agent with several immediate options of 
unknown value can decide what to do by first examining future actions that 
eventually lead to states of known value.} The process of looking for a 
sequence of actions that reaches the goal is called \textbf{search}. A search
algorithm takes a problem as input and returns a \textbf{solution} in the 
form of an action sequence. Once a solution is found, the actions it recommends
can be carried out. This is called the \textbf{execution} phase. Note while
the agent is executing, it \emph{ignores its percepts} when choosing its actions
because it knows in advance what they will be.\\

Together, the \textbf{initial state, actions and transition model} implicitly
define the \textbf{state space} of the problem - the set of all states
reachable from the initial state by any sequence of actions. A \textbf{solution}
to a problem is an action sequence that leads from the initial state to a goal
state. Solution quality is measured by the path cost function, and an \textbf{
optimal solution} has the lowest path cost among all solutions. The process 
of removing detail from a representation is called \textbf{abstraction}.\\

It is quite important to distinguish between a node and a state: a node is a 
bookkeeping data structure used to represent the search tree. A state 
corresponds to a configuration of the world. Furthermore, two different states
can contain the same world state if that state is generated via two different
search paths. We can evaluate the performance of an algorithm in four ways:
\begin{itemize}
    \item \textbf{Completeness}: Is the algorithms guaranteed to find a solution
    if there is one?
    \item \textbf{Optimality}: Does the strategy find the optimal solution?
    \item \textbf{Time complexity}: How long does it take to find a solution?
    \item \textbf{Space complexity}: How much memory is needed to perform the
    search?
\end{itemize}
In AI, the graph is often represented implicitly by the initial state, actions
and transition model and is frequently infinite. Complexity is expressed in
terms of three quantities: \emph{b}, the \textbf{branching factor} or maximum
number of successors of any node; \emph{d}, the \textbf{depth} of the shallowest
goal node; and \emph{m}, the maximum length of any path in the state space.

\subsection{Uniformed Search Strategies}
\subsubsection{Breadth-First Search}

\end{document}